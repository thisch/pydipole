% execute it with pdflatex -shell-escape equations.tex
\documentclass[multi={mymath},border=1pt,convert={density=300,outext=.png}]{standalone}
\usepackage{amsmath}
\newenvironment{mymath}{$\displaystyle}{$}
\usepackage{xcolor}
\begin{document}

\pagecolor[RGB]{255,255,255} % white

% \newcommand{\norm}[1]{\left\lVert#1\right\rVert}
\newcommand{\norm}[1]{\left|#1\right|}

\begin{mymath}
\mathbf{E}(\mathbf{r}, t) = \sum_{n=1}^{N}\frac{1}{4\pi\epsilon_{0}}\left\{ \frac{k^{2}}{r'}(\mathbf{\hat{r}'} \times \mathbf{p}_{n}) \times \mathbf{\hat{r}'} + \frac{1}{r'^{3}}(1-ikr')\left[3\mathbf{\hat{r}'}(\mathbf{\hat{r}'}\cdot\mathbf{p}) - \mathbf{p}\right]\right\}\,e^{ikr' - i(\omega t + \varphi_{n})}
\end{mymath}

% definitions
\begin{mymath}
r' := |\mathbf{r} - \mathbf{a}_{n}|\quad\mathbf{\hat{r}'} := \frac{\mathbf{r} - \mathbf{a}_{n}}{|\mathbf{r} - \mathbf{a}_{n}|}
\end{mymath}

% farfield of electric field
\begin{mymath}
\mathbf{E}_{\infty}(\mathbf{r}, t) = \sum_{n=1}^{N}\frac{k^{2}}{4\pi\epsilon_{0}r} (\mathbf{\hat{r}} \times \mathbf{p}_{n}) \times \mathbf{\hat{r}}\,e^{ik(r - \mathbf{\hat{r}}\cdot\mathbf{a}_{n}) - i(\omega t + \varphi_{n})}
\end{mymath}

% radiant intensity (radiant flux per unit solid angle)
\begin{mymath}
  \frac{\partial \overline{P}}{\partial \Omega} = r^{2} |\langle S_{r} \rangle| = \frac{r^{2}}{2Z} \norm{\mathbf{E}_{\infty}}^{2} = \frac{k^{4}}{32\pi^{2}\epsilon_{0}^{2}Z} \norm{\sum_{n=1}^{N} (\mathbf{\hat{r}} \times \mathbf{p}_{n}) \times \mathbf{\hat{r}}\, e^{- i(k\mathbf{\hat{r}}\cdot\mathbf{a}_{n} + \varphi_{n})}}^{2}
\end{mymath}

\end{document}
%%% Local Variables:
%%% mode: latex
%%% TeX-master: t
%%% compile-command: "pdflatex -shell-escape equations.tex"
%%% End:
